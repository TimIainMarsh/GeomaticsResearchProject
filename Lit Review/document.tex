\documentclass[11pt,a4paper,final]{article}
\usepackage[latin1]{inputenc}

\usepackage{amsmath}
\usepackage{amsfonts}
\usepackage{amssymb}
\usepackage{graphicx}
\usepackage{float}



\author{Tim Marsh}
\title{Geomatics Research Project Proposal}
\date{11 March 2015}


\bibliographystyle{plain}

\begin{document}
	
	\pagebreak
	\tableofcontents
	\pagebreak
	
	\section{Literature Review}
	
	With how most of the world these days is online all the time, things such as Google Maps and Apple Maps are hugely popular as ways of navigating around. and because Internet connections are so fast the maps we can view with these applications are becoming more and more detailed. so with the outside world so well mapped and available to the public the next logical progression is to move indoors.
	
	But with conventional methods of creating models of buildings being very slow or not accurate, the rate at which we are going to see navigate-able indoor models become popular will be slow.
	
	So having the ability to create indoor models quickly and accurately will speed up the process of having everyone able to access indoor models.
	
	Having easily attainable indoor models is not only something people want as a fun was of looking around buildings, it also has serious commercial uses in architecture and engineering.
	
	\setcounter{page}{1}
	\pagenumbering{arabic}
	

	
\end{document}

	%Lit Review-A review of the relevant literature-This should not simply be a list of summaries with some comments added-on, but an integrated statement that explains why these studies or theories-are important to your research. (See what we said earlier about "present a-case" and "justify what you plan to do"!)
	
	
	
