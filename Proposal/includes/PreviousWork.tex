

\section{Previous Work}

% % Not this 

There seem to be no previous examples of this project in previous papers, but there are two papers that are relevant in one way or another to this project.

The first is a paper that was written in 2012 by the image Processing Lab at the University of California that present an automatic system for planar 3D modelling of building interiors from point cloud data generated by range scanners \cite{VS12}. This paper is very relevant because it is more or less the same topic that this paper proposes. It however does not look at wire frames instead it looks at an automated 3D modelling process.

Their process looks at a 3 step system, the first step is a Principal Components Analysis that gives each point a normal.
It then classifies and segments the points using the normals, for example a point with a normal pointing downwards is likely to belong to the celling. So they classify the point as ceiling and then create segments by grouping similarly classed points together.
Once the points are classed and segmented the segments then have a model fitted tot them. the outline planar modelling and modelling of a staircase. In the results section they state that they used C++ and PCL (Point Cloud Library). Their future plan is to create a process that can model non planar systems and look at curved walls.\\

% % More likely this
% % but not exactly

The paper mentioned above dealt with creating 3D planer models form point clouds, another option is to create wire-frame models of the point clouds. This is done in a 2010 paper by Karim Hammoudi and others\cite{Ham12}. The paper titled Extracting wire-frame models of street fa\c{c}ades from 3D point clouds and the corresponding cadastral map. The main difference with this and the above paper is that it is creating wire frame models as opposed to 3D models, and is of the exteriors of buildings not the inside.

After the raw 3D point cloud has been filtered, it is segmented into a cloud of only the fa\c{c}ade of the buildings on the street then estimates of the each individual fa\c{c}ades are extracted using the cadastral map. A Progressive Probabilistic Hough Transform (PPHT) is used to determine relationships between points and create the wire frame model.



