%Introduction
%-Nature of the problem
%-Why the problem is important
%-How your research would contribute to the solution of the problem

\chapter{Introduction}

%The reconstruction of precise surfaces from unorganized
%point clouds derived from laser scanner data or
%photogrammetric image measurements is a very hard
%problem, not completely solved and problematic in case of
%incomplete, noisy and sparse data.


3D representations of buildings vary largely in terms of how rigorously the models are structured. There are two extremes to it, the first is highly structured models, made in CAD software, of the buildings before or after the building has been built. These models are made by people using measurements.

The other end of the scale is 3D points clouds created by laser scanners that are entirely unstructured, this is because they are just points floating in a 3D space. There is no relation from one point to another.

For taking surveying related measurements having just a point cloud works fine, you can pick out specific points and do measurements across the points. But when 3D models are used for more complex applications such as an as in engineering and architecture, just having a point cloud will not cut it. These applications need structure that a point cloud can produce.

So it becomes necessary to create structure in these 3D points clouds, or to create relationships between these point. This is usually done by turning the point cloud into 3D models that engineers and architects can then work on and use. There is an added advantage that some methods used for point cloud segmentation work with by classifying the points which allows for the 3D model to have labels.



This paper goes through a method of creating wire-frame models from point clouds taken with laser scanners in enclosed spaces such as rooms and halls. 


\section{Methodology}

\subsection{Objectives}

The objective of this project is to be able to create wire frame models of the interior of buildings.

Point clouds are notoriously difficult to navigate on computers, especially the interior of buildings. creating a wire frame model makes the buildings interior easier to navigate, measure and use in a practical way.

The goal of surface reconstruction can be said as follows: Given a set of sample points P assumed to lie on or near an unknown surface S, create a surface model S' approximating S.
A surface reconstruction procedure cannot guarantee the recovering of the surface exactly, since we have information about the surface only through a finite set of points.

The programming will be done in C++ as this language is popular, has a large community and supports many large and popular libraries such as Point Cloud Library (PCL). 

Using C++ and PCL an algorithm to complete these tasks with as little intervention as possible will be written as an add on function onto existing software.

For this to take place all outputted data must be in recognised file formats.


\subsection{Questions}
\begin{itemize}
	
	\item How do we go about segmenting the point cloud?
	
	\item What is the best way to find breaklines, edges?
	
	\item How do we decide which point belongs to which room in the case of multiple rooms?
	
	\item Generalising, removing points? adding points if necessary?
	
	\item Triangulation, use all points? or minimize number of lines in the render?
	
	\item RANSAC?
	
\end{itemize}


\subsection{Proposed methods}


From points to surface:
%section 4.2 from FROM POINT CLOUD TO SURFACE: THE MODELING AND VISUALIZATION PROBLEM

\begin{enumerate}
	\item Pre-Processing - Removing noise, cleaning up data to allow work to be done on it, or sampling to reduce computation time.
	
	\item Determination of the global topology of the objects surface - Look for segments and make sure features such as breadlines and such are preserved
	
	\item Generation of the surface - Triangular meshes or planes are created satisfying certain requirements e.g. size limits ect \ldots
	
	\item Post processing - When the model is created, editing operations are commonly applied to refine and perfect the polygonal surface
\end{enumerate}

\section{Other Considerations}

In some cases there may be an shortage of points or in the more likely case there are to many points. for example a wall, we don't need 100 000 points to approximate a wall, so we can generalize that segment to make computations easier and faster.

But equally we may get a terrible approximation if we are missing large portions of the wall due to scan shadows we may get a really bad approximation of the wall in places so adding points may become necessary.



\section{Outcomes}

The outcome of this paper will be to create an automatic system that creates 3D wire-frame models of the interior of a room/rooms in a building. 

So if a large point cloud that represents the whole interior of a building is fed through the program it will return a 3D wire-frame model of the interior of that building.qqq

The program will be created in C++ using Point Cloud Library (PCL) and then be used as an add on function for existing point cloud software.







