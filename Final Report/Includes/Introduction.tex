%Introduction
%-Nature of the problem
%-Why the problem is important
%-How your research would contribute to the solution of the problem

\chapter{Introduction}

%The reconstruction of precise surfaces from unorganized
%point clouds derived from laser scanner data or
%photogrammetric image measurements is a very hard
%problem, not completely solved and problematic in case of
%incomplete, noisy and sparse data.

POINTS $--->$ MODEL\\
\\

3D representations of buildings vary largely in terms of how rigorously the models are structured. There are two extremes to it, the first is highly structured models, made in CAD software, of the buildings before or after the building has been built. These models are made by people using measurements.

The other end of the scale is 3D points clouds created by laser scanners that are entirely unstructured, this is because they are just points floating in a 3D space. There is no relation from one point to another.

For taking surveying related measurements having just a point cloud works fine, you can pick out specific points and do measurements across the points. But when 3D models are used for more complex applications such as an as in engineering and architecture, just having a point cloud will not cut it. These applications need structure that a point cloud can produce.

So it becomes necessary to create structure in these 3D points clouds, or to create relationships between these point. This is usually done by turning the point cloud into 3D models that engineers and architects can then work on and use. There is an added advantage that some methods used for point cloud segmentation work with by classifying the points which allows for the 3D model to have labels.



\section{Objectives}

The objective of this project is to be able to create boundary representation models from laser scans of an indoor environment.

Point clouds are notoriously difficult to navigate on computers, especially the interior of buildings. creating a wire frame model makes the buildings interior easier to navigate, measure and use in a practical way.

The goal of surface reconstruction can be said as follows: Given a set of sample points P assumed to lie on or near an unknown surface S, create a surface model S' approximating S.

A surface reconstruction procedure cannot guarantee the recovering of the surface exactly, since we have information about the surface only through a finite set of points.

\section{Implementation}



The programming will be done in C++ as this language is popular , has a large community and supports many large and popular libraries such as Point Cloud Library (PCL). 

Using C++ and PCL an algorithm to complete these tasks with as little intervention as possible will be written as an add on function onto existing software.




\section{Research Questions}
\begin{itemize}
	
	\item What is the best and most efficient method to segment a point cloud?
	
	\item What is the best method of getting model coefficients of planar segments?
	
	\item (Combine surface to create models)
	
	\item What is the most effective boundary representation storage method?
	
\end{itemize}


\section{Other Considerations}

-sub sample

-speed

-output file

-




\section{Outcomes}

The outcome of this paper will be to create an automatic system that creates 3D boundary representation models from laser scans of rooms. 

The program will be created in C++ using Point Cloud Library (PCL) and then be used as an add on function for existing point cloud software.







