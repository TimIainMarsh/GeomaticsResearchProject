
\chapter{Method}
	
	Points within a 3d point cloud are simply represented by their x, y, z Cartesian coordinates. This means that there is a possibility that there are two points with the same x, y, z coordinates that have been acquired at different times, this is assuming that the origin is the same. Comparing these points is not as easy as one would assume, they may occupy the same point in space, but cannot simply assume that they are the same.
	
	We may get some extra data from a laser scanner, such as intensity and colour. But this doesn't really give us any information about the area surrounding these points and weather anything has changed.
	
	Situations where points need to be compared for any reason require more information than can be provided by a laser scanner. So the idea of looking at each point individually now falls away. We need to start looking at the bigger picture of what the point cloud is telling us.\\
	\\
	Machine learning is a subfield of computer science that evolved from pattern recognition and learning theory in artificial intelligence. it is in the field of machine learning that we start to look at the bigger picture of what our point clouds are telling us. Machine learning is the study of creating algorithms that can learn from and make predictions about data.\\
	\\
	The aim of this Thesis is, given an uncleaned, noisy point cloud of a room to create an .obj file with lines representing all the edges of that room.
	
	
\section{Segmentation}

	\subsection{Principal Components Analysis}
		A Principal Components Analysis (PCA) is the default method for estimating normals in Point Cloud Library. it is the fastest method because of PCL's multi-threading option for the function.
		
		A principal components analysis can be thought of as fitting an \textit{n}-dimensional ellipse to a set of data. Each axis of the ellipse represents a principal component. If an axis is large then the variance along that axis is large, and vice versa. When calculating the normal of a point set, the smallest axis is the axis with with the least variance and therefore represents the normal. This is easy to think of with a 2D disk of points, like in figure \ref{fig:PCA2D Example}. Its easy to see how the 3rd axis for these set of point will be coming out of the page, as it has to be orthogonal to the other two.
		
		\begin{figure}[H]
			\centering
			\begin{subfigure}{.5\textwidth}
				\centering
				\includegraphics[width=1\linewidth]{Includes/images/pca1}

				\label{fig:sub1}
			\end{subfigure}%
			\begin{subfigure}{.5\textwidth}
				\centering
				\includegraphics[width=1\linewidth]{Includes/images/pca2-ver2}

				\label{fig:sub2}
			\end{subfigure}
			\caption{A 2D representation of a Principal Components Analysis }
			\label{fig:PCA2D Example}
		\end{figure} 
		
		To fit this \textit{n}-dimensional ellipse we compute the covariance matrix of the data set and calculate the eigenvalues and corresponding eigenvectors of this covariance matrix.
		
		\subsubsection{Calculating the Covariance Matrix of a given set of 3D points}
			The covariance matrix, $C$, is calculated for each point $p_i$ as follows:
			
			\begin{equation}
			C = \frac{1}{k} \sum_{i=1}^{k}.(p_i - \bar{p}).(p_i - \bar{p})^T
			\end{equation}
			
			Where $k$ is the number of points in the neighborhood of point $p_i$, and $\bar{p}$ is the 3D centroid of the nearest neighbors.
				
		\subsubsection{Calculating Eigenvalues and Eigenvectors}
			Eigenvalues, $\lambda_j$, and eigenvectors, $\vec{v_j}$, are calculated on the covariance matrix, $C$, for a given point:
			
			\begin{equation}
			C \vec{v_j} = \lambda_j \vec{v_j}
			\end{equation}
			After solving the system and getting results for $\lambda_j$ and $\vec{v_j}$, the smallest eigenvalue and its corresponding eigenvector is selected as the normal to the point set.\\
			\\
			\\
			There is no mathematical method for determining the sign of the normal, its orientation as computed in the PCA is ambiguous, and not consistently orientated over the whole point cloud. 
			
			The solution to this issue is simple. If the viewpoint is known, in the case of laser scans it is as the scan center, then orientate all normals $\vec{n_i}$ towards the viewpoint.

				
	\subsection{Region Growing}
	%mention colour based region grwoing
	
		Point Cloud Libraries region growing algorithm starts off by calculating the curvature for each point then sorting all points by their curvature value. this is because the point with the least curvature associated with it is located in the flattest section of the point cloud. Once the cloud is sorted the region growing part of the algorithm begins:
		
		\begin{itemize}
			\item The algorithm picks the point with the smallest curvature and adds it to a set called seeds.
			
			\item For every seed point the algorithm looks at all the neighbors and decides if they are part of that seeds region or not. This is done by looking at certain criteria with user specified thresholds.
			
			\begin{itemize}
				\item Does the points normal deviate by more than the user specified threshold from the seed point.
				
				\item Does the points curvature deviate by more than the user specified threshold from the seed point.
				
			\end{itemize}
			
			\item Once no more points are found for that particular seed, or the region reached a specified maximum, the seed is removed from the set of seeds and the region as added to the global segment list. If no more seeds are in the set the process is started from the beginning again, except this time the points that have been assigned to a region are no longer available to be seed points or to become part of a region.
			
		\end{itemize}
		
		If after a seed point has been through the process and the number of points does not reach the user defined minimum size the whole region removed from the cloud as unclassified and will inevitably be removed by the user.
		
	


\section{Surface Extraction}
		
		\subsection{Segment selection}
			Vert
			Size
			
			Hor

		\subsection{RANSAC}


\section{Model Generation}
		\subsection{Bounding Box Generation}
		OBB
		Moment of inertia
		
		\subsection{Angle between Planes}
		
		\subsection{Plane with Plane Intersection}
		
		\subsection{Orthogonal Distance of point to line}
		
		\subsection{Project points onto line}
		
		\subsection{Create OBJ file}

	


