\section{Method}


	
	
	% computer vision
	
	Points within a 3d point cloud are simply represented by their x, y, z Cartesian coordinates, with respect to a given origin. This means that there is a possibility that there are two points with the same x, y, z Cartesian coordinates that have been acquired at different times, this is assuming that the origin is the same. Comparing these points is not as easy as one would assume, they may occupy the same point in space, but who are we do assume that they are the same.
	
	We may get some extra data from a laser scanner, such as intensity and colour. But this doesn't really give us any more information about the area surrounding these points and weather anything has changed.
	
	Situations where points need to be compared for any reason require more information than can be provided by a laser scanner. So the idea of looking at each point individually now falls away. We need to start looking at the bigger picture of what the point cloud is telling us.
	

	

	
	
	\subsection{Region Growing}
	
	\subsection{Plane Fitting to Segments}
	
	\subsection{Removal of Segments}
	
	\subsection{Intersection of Planes}
	
	\subsection{title}