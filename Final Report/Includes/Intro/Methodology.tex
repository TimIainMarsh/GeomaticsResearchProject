\subsection{Methodology}

\subsubsection{Objectives}

The objective of this project is to be able to create wire frame models of the interior of buildings.

Point clouds are notoriously difficult to navigate on computers, especially the interior of buildings. creating a wire frame model makes the buildings interior easier to navigate, measure and use in a practical way.

The goal of surface reconstruction can be said as follows: Given a set of sample points P assumed to lie on or near an unknown surface S, create a surface model S' approximating S.
A surface reconstruction procedure cannot guarantee the recovering of the surface exactly, since we have information about the surface only through a finite set of points.

The programming will be done in C++ as this language is popular, has a large community and supports many large and popular libraries such as Point Cloud Library (PCL). 

Using C++ and PCL an algorithm to complete these tasks with as little intervention as possible will be written as an add on function onto existing software.

For this to take place all outputted data must be in recognised file formats.


\subsubsection{Questions}
\begin{itemize}

\item How do we go about segmenting the point cloud?

\item What is the best way to find breaklines, edges?

\item How do we decide which point belongs to which room in the case of multiple rooms?

\item Generalising, removing points? adding points if necessary?

\item Triangulation, use all points? or minimize number of lines in the render?

\item RANSAC?

\end{itemize}


\subsubsection{Proposed methods}


From points to surface:
%section 4.2 from FROM POINT CLOUD TO SURFACE: THE MODELING AND VISUALIZATION PROBLEM

\begin{enumerate}
\item Pre-Processing - Removing noise, cleaning up data to allow work to be done on it, or sampling to reduce computation time.

\item Determination of the global topology of the objects surface - Look for segments and make sure features such as breadlines and such are preserved
	
\item Generation of the surface - Triangular meshes or planes are created satisfying certain requirements e.g. size limits ect \ldots

\item Post processing - When the model is created, editing operations are commonly applied to refine and perfect the polygonal surface
\end{enumerate}

\subsubsection{Other Considerations}

In some cases there may be an shortage of points or in the more likely case there are to many points. for example a wall, we don't need 100 000 points to approximate a wall, so we can generalize that segment to make computations easier and faster.

But equally we may get a terrible approximation if we are missing large portions of the wall due to scan shadows we may get a really bad approximation of the wall in places so adding points may become necessary.



