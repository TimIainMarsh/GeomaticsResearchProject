%\addcontentsline{toc}{section}{Introduction}

%Introduction
%-Nature of the problem
%-Why the problem is important
%-How your research would contribute to the solution of the problem

\section{Introduction}

%The reconstruction of precise surfaces from unorganized
%point clouds derived from laser scanner data or
%photogrammetric image measurements is a very hard
%problem, not completely solved and problematic in case of
%incomplete, noisy and sparse data.


3D representations of buildings vary largely in terms of how rigorously the models are structured. There are two extremes to it, the first is highly structured models, made in CAD software, of the buildings before or after the building has been built. These models are made by people from scratch using measurements.

The other end of the scale is 3D points clouds created by laser scanners that are entirely unstructured, this is because they are just points floating in a 3D space. There is no relation from one point to another.

For taking surveying related measurements having just a point cloud works fine, you can pick out specific points and do measurements across the points. But when point clouds are used for more complex applications such as an as-built survey of a factory with the intention of the factory being expanded and/or up graded, only a point cloud wont cut it.

So it becomes necessary to create structure in these 3D points clouds, or to create relationships between these point. This is usually done by turning the point cloud into 3D models that engineers can then work on and use. There is an added advantage that some methods used for point cloud segmentation work with by classifying the points which allows for the 3D model to have labels.

This paper goes through a method of creating wire-frame models from point clouds taken with laser scanners in enclosed spaces such as rooms and halls. 








